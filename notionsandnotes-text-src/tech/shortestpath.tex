
\begin{frame}[t,plain]
\titlepage
\end{frame}

\begin{frame}[t]{BLE Routing Parameters}

Take note:

\begin{itemize}
\item Each device knows only about its neighbors.
\begin{itemize}
\item So, centralized algorithms for routing are not possible.  
\item We assume a method exists for neighbor discovery.
\end{itemize}
 
\item Each device has low memory; in total only 64kB is available. Therefore,  
\begin{itemize}
\item We consider for the moment only a limited number of devices in total. 
\item Schemes such as interval routing, PASTRY etc. can be investigated later to minimize memory consumption.  
\end{itemize}
\item Network topology is liable to change. 
\begin{itemize}
\item So algorithm needs to be adaptable to changes in topology. 
\item But network is mostly fixed and changes happen only once in a while.  
\end{itemize}  
\item We only aim at a workable prelimary algorithm for a hundred or so devices.  
\item Again, we consider a workable algorithm, and not the optimal one.  
\end{itemize}
\end{frame}

\begin{frame}{Distance Vector Routing}

\begin{block}{Question}
Each node knows only about its neighbors. With this information we need a method to route any packet to any destination.
\end{block}
  
\begin{block}{Solution : Distance Vector Routing.}
Patch up the local information using a distributed algorithm to determine a global routing table.
 

\begin{itemize}
\item Each node maintains a table of distances to every other node
\item This information is spread to its neighbors whenever required
\item Using such updates, the distance table is improved at each turn
\item The algorithm converges eventually to actual distances/routes.
\end{itemize}
\end{block}
\end{frame}

\begin{frame}{DVR : Begin to formalize}
Let us suppose we have a directed graph $G = (V, E)$ with vertex set $V$ consisting of $|V| = N$ vertices and edge set $E$. Weight of an edge $e$
is denoted $wt(e)$.

Each node $I$ has a distance vector or array consisting of $N$ entries, viz., $\delta (I,0), \delta (I, 1), \ldots , \delta (I,N-1)$. $\delta (I,J)$ is the distance from node $I$ to $J$. Initialize all $\delta (I,I) = 0$ and all $\delta (I,J) = \infty$ when $I \neq J$.
 
\begin{block}{Idea : Relaxation}
\begin{itemize}
\item Relax the requirements.
\item Algorithm finds a solution to the easier problem.
\item Algorithm tightens the constraints in some part and improves the solution. Repeat.
\end{itemize}
\end{block}
 
\vskip -.7em
\begin{block}{Application}
Find suboptimal distances and paths. Improve repeatedly on them. Converge to best solution.
\end{block}
\end{frame}

\begin{frame}{Example : Shiju's Graph}
\begin{tikzpicture}[->,>=stealth',shorten >=1pt,auto,node distance=3cm,
  thick,main node/.style={circle,fill=blue!20,draw,font=\sffamily\Large\bfseries}]

  \node[main node] (A) {A};
  \node[main node] (B) [above right of=A] {B};
  \node[main node] (F) [below right of=B] {F};
  \node[main node] (C) [above right of=F] {C};
  \node[main node] (D) [below right of=C] {D};
  \node[main node] (E) [below of=F] {E};

  \path[every node/.style={font=\sffamily\small}]
    (A) edge node {1} (F)
    (A) edge node {1} (B)
    (B) edge node {1} (C)
    (B) edge node {1} (D)
    (B) edge node {1} (F)
    (C) edge node {1} (D)
    (D) edge node {1} (E)
    (E) edge node {1} (A)
    (F) edge [bend right] node {1} (C)
    (F) edge node {1} (D)
    (F) edge node {1} (E);
\end{tikzpicture}
\end{frame}

\begin{frame}{Centralized Bellman-Ford(Relaxation)}

Assume that we know the topology of the whole graph. For simplicity, there is at most one link from any node $I$ to another node $J$.

\begin{itemize}
\item We need to construct a shortest path tree with root at $A$.
\item For each node, it is enough to know its parent in this tree.
\end{itemize}
The algorithm runs as follows:
\begin{enumerate}
\item Initialize all distances as $\delta (I,I) =0$ and $\delta (I,J) = \infty$ for $I\neq J$.
\item For each node $I$, its parent $p(I)$ is initially set to \texttt{NULL}.
\item Order the set of edges in some manner as $E_0, E_1, E_2, \ldots , $.
\item Let $i$ run from $0$ to $|V| -1$ where $|V|$ is the cardinality of the vertex set. Then,
\begin{itemize}
\item Let $e$ run from $E_0$ through $E_1, E_2, \ldots,$. Let $e = (X,Y)$ be connecting node $X$ to $Y$. Then, 
\begin{itemize}
\item[] If $\delta (A, Y) > \delta (A,X) + wt(e)$, Then,
\item[] \qquad set $\delta (A,Y)= \delta (A,X)+wt(e)$ 
\item[] \qquad set $p(Y) = X$.
\end{itemize} 
\end{itemize}
\end{enumerate}

\end{frame}

\begin{frame}{Remarks}
It can be proved that the algorithm converges to correct distances.
\vskip 2em
The time complexity is of order $\mathcal{O}(|V||E|)$ where $|V|$ and $|E|$ are cardinalities of vertex and edge sets.
\vskip 2em
We assumed that the whole network topology is known. Bellman - Ford is a distance vector routing method. If whole network
topology is known on a processor with good resources, then Dijkstra's algorithm or Chandy-Misra's algorithm are better.
\end{frame}

\begin{frame}{Asynchronous Bellman-Ford : Fixed source}

We need to construct a shortest path tree with root at $A$ and for each node, it is enough to know its parent in this tree.
Initialize as follows:
\begin{itemize}
\item Initialize all distances as $\delta (I,I) =0$ and $\delta (I,J) = \infty$ for $I\neq J$.
\item Node $A$ sends $\delta (A,I)=wt(A,I)$ to each neighbor $I$ of $A$ as a pair $(wt(A,I), I)$.
\end{itemize}
Run the following process at each node $I$. $\delta (A,I)$ is simply denoted $\delta (I)$.
\begin{itemize}
\item Wait for messages of format $(S, J)$. Ignore if $S \geq \delta (I)$. Otherwise,
\begin{itemize}
\item[] If parent $p(I) \neq J$ then set $p(I) = J$.
\item[] Set $\delta (I) = S$.
\item[] Send pair $(\delta (I) + wt(I,K), I)$ to all neighbors $K \neq p(I)$.
\end{itemize}
\end{itemize}
\end{frame}


\begin{frame}{Asynchronous Bellman-Ford : All Pairs}

\begin{block}{Neighbor Distance Data}
\begin{itemize}
\item[] Maintain two tables at each node $I$, viz, distance table(=routing table), and neighbor table,
with distances via each neighbor to all nodes.
\item[] The degree of the graph, i. e., number of neighbors for each node, is assumed have an upper bound $d$.
\item[] So memory requirement at each node is less than $(d+1)n$ where $n$ is the number of nodes.
\item[] It is enough to store only distance and the next hop towards each destination in some modifications of the algorithm, with other tradeoffs.
\end{itemize}
\end{block}
\end{frame}

\begin{frame}{Asynchronous B-F Algorithm (Contd.)}
\begin{block}{Basic idea}
\begin{itemize}
\item Each node $I$ sends its distance table $\delta(I,\_)$ to all its neighbors whenever it is updated.
\item Each node $X$, upon reception of such an update, recomputes its own distace table as follows: 
\item[] \begin{itemize}\item[] For destination $Y$, distance $\delta(X, Y)$ is set to minimum over all neighbors 
$V$ of $X$ of: $\delta(V,Y) + wt(X,V)$. \item[] Here $wt(X,V)$ is the weight of the edge connecting $X$ to $V$. If there is more than one such edge, we
take minimum again. \end{itemize}
\end{itemize}
\end{block}
\end{frame}

\begin{frame}{Asynchronous B-F , Procedure, All- Pairs}
Start same routine at all nodes $I$, with slight randomness and retransmission to alleviate collision issues.

$\delta(I,J,K)$ denotes distance from $I$ to $K$ with $J$ as the next hop, where $J$ has to be a neighbor of $I$.
\begin{block}{Initialize}
\begin{itemize}
\item For all neighbors $J$, and for all nodes $K \neq J$, set $\delta(I,J,K) =\infty$.
\item Set $\delta(I,J,J) = wt(I,J)$.
\item For each $K$, compute $\delta(I,K)$ as the minimum of $\delta(I,J,K)$ as $J$ runs through all neighbors of $I$. Broadcast this value to all neighbors of $I$.
\end{itemize}
\end{block} 
\end{frame}

\begin{frame}{Asy. B-F , Procedure, All- Pairs, Contd.}
Node $I$ waits for, either: the weight of a link to a neighbor $J$ to change, in which case:
\begin{block}{Link updation}
\begin{itemize}
\item[] If $wt(I,J)$ changes by $\partial$, then for all $K$, reset $\delta(I,J,K) = \delta(I,J,K) + \partial$.
\end{itemize}
\end{block} 
or, for a distance table update of $\delta^{new}(J,K)$ towards destination $K$ from a neighbor $J$,
\begin{block}{Neigbhor distance updation}
\begin{itemize}
\item[] Set $d(I,J,K) = \delta^{new}(J,K) + wt(I,J)$  and update $\delta(I,K)$ if necessary.
\end{itemize}
\end{block} 
In either case, send the updated distance table values to all neighbors, and wait for either event as above, forever. 
\end{frame}

\begin{frame}{Remarks}
\begin{itemize}
\item Good part : Network flow is maximized under certain conditions(Cf, Ford \& Fulkerson) 
\item Good part : Routing takes places even before full convergence
\item Bad part : Difficult to optimize convergence. Sometimes does not converge at all upon deletion of links.
\item Difficulty: Asynchronous distributed processing is an also art.
\item Worst case behavior exponential, $\mathcal{O}(2^n)$. In practice rarely so bad.
\item Have to experiment and find out constraints for good behavior.
\item Such determiniation best done in simulation.
\item Probabilistic estimates of messages of per unit time are of the order of $nd^{2+1/m}(\mathrm{ln}\ d)^{1+1/m}$, where $n$ is number of nodes, $d$ is the max. degree of a node, and $m$ is a constant depending on probability distribution used.
\end{itemize}
\end{frame}

\begin{frame}{Addition, Deletion and Movement of links}

\begin{block}{Running}
The algorithm runs always, and asynchronously, at all nodes. Algorithms connects with all nodes within reach, and then within their reach, etc..
\end{block}

\begin{block}{Offline/Online nodes}
If a node is offline, distances to that node from all nodes will be $\infty$. 
We generally consider links(rather than nodes) going up and down, and these are reflected in their weights
being finite or infinite. 
\end{block}

\begin{block}{Movement of nodes}
When a node is moved, it is considered that some links from that node go down, and some other links come up. Therefore the problem
of node movement is reduced that of addition and deletion of links.
\end{block}
\end{frame}

\begin{frame}{Contd.}

\begin{block}{Upon startup, or addition of node/link}
If a node $X$ gets within range of a network with converging or converging Bellman-Ford algorithm, then:
\begin{itemize}
\item Some distances to $X$ will become finite.
\item This information gets passed around, and new distances are computed, and passed around.
\item B-F converges to newly optimal distances involving $X$ too.
\end{itemize}
\end{block}

\begin{block}{Good news and Bad news}
\vskip 1em
General paradigm:
\begin{itemize}
\item Whenever a link goes up, or edge weights are reduced, algorithm updates new routes and distances fast.
\item Propagation of information of link failure is slower. 
\end{itemize}
\end{block}
\end{frame}

\begin{frame}{Deletion of links}
\begin{block}{Problem}
Upon deletion of a link, sometimes algorithm converges successfully. In certain other scenarios, it results in problems. 
Examples are:
\begin{itemize}
\item Looping
\item Bouncing
\item Counting to infinity
\end{itemize}
\end{block}

\begin{block}{Solution}
The Bellman-Ford algorithm is modified by a number of people in a number of situations to solve the problems. Two popular methods are:
\begin{itemize}
\item Set upper bounds for distances to avoid counting to infinity.
\item Split horizon : Do not inform a neighbor of paths through that neighbor.
\end{itemize}
\end{block}
\end{frame}

\begin{frame}{Contd.}
\begin{block}{Solution: Contd.}
\begin{itemize}
\item Poisoned reverse: If a path from $I$ to $A$ goes through a neighbor $X$, tell $X$ that distance $\delta (I,A) = \infty$.
\item Assign time to live(TTL) in routing protocol.
\item Hold down: Increase delays as number of hops go up.
\item Probablistic methods.
\item Use a distributed version of the Dijkstra algorithm, instead of B-F,
\item Have a "router" with higher processing abilities, and run the Dijkstra or Chandy-Misra algorithms there.
\item etc.
\end{itemize}
\end{block}
Each method has pros and cons. This is an issue to be examined closer and decided.
\end{frame}


\begin{frame}{Illustration : B-F Convergence}
Initialization:
\begin{columns}
\column{.45\textwidth}
\begin{tikzpicture}[>=stealth',shorten >=2pt,auto,node distance=1.5cm,
  thin,main node/.style={circle,fill=blue!20,draw,font=\sffamily \bfseries}]


  \node[main node] (A) {A};
  \node[main node] (B) [above right of=A] {B};
  \node[main node] (F) [below right of=B] {F};
  \node[main node] (C) [above right of=F] {C};
  \node[main node] (D) [below right of=C] {D};
  \node[main node] (E) [below of=F] {E};

  \path[every node/.style={font=\sffamily\small},thin]
    (A) edge (F)
        edge (B)
    (B) edge (C)
        edge (D)
        edge (F)
    (C) edge (D)
    (D) edge (E)
    (E) edge (A)
    (F) edge (C)
        edge (D)
        edge (E);


\end{tikzpicture}

\column{.55\textwidth}
\begin{table}[h]
\begin{tabular}{|r|c|c|c|c|c|c|}
\hline
 & \multicolumn{6}{c|}{Distance $\delta (\_,J)$ to $J$} \\ 
\hline
$\delta (I,\_)$ & A &  B& C & D & E  & F \\
\hline
A & 0 & $\infty$ & $\infty$ & $\infty$ & $\infty$ & $\infty$ \\
\hline
B & $\infty$ & 0 & $\infty$ & $\infty$ & $\infty$ & $\infty$ \\
\hline
C & $\infty$ & $\infty$ & 0 & $\infty$ & $\infty$ & $\infty$ \\
\hline
D & $\infty$ & $\infty$ & $\infty$ & 0 & $\infty$ & $\infty$ \\
\hline
E & $\infty$ & $\infty$ & $\infty$ & $\infty$ &0  & $\infty$ \\
\hline
F & $\infty$ & $\infty$ & $\infty$ & $\infty$ & $\infty$ & 0 \\
\hline
\end{tabular}
\end{table}
\end{columns}
Assume all bidirectional links with weight $1$.
\end{frame}


\begin{frame}{Illustration : B-F Convergence}
Suppose $A, B, C, D ,E$ are up and $F$ is off. Suppose B-F process starts at $A$ and $D$ simultaneously and updates link info.
\begin{columns}
\column{.45\textwidth}
\begin{tikzpicture}[>=stealth',shorten >=2pt,auto,node distance=1.5cm,
  thin,main node/.style={circle,fill=blue!20,draw,font=\sffamily \bfseries}]


  \node[main node] (A) {A};
  \node[main node] (B) [above right of=A] {B};
  \node[main node] (F) [below right of=B] {F};
  \node[main node] (C) [above right of=F] {C};
  \node[main node] (D) [below right of=C] {D};
  \node[main node] (E) [below of=F] {E};

  \path[every node/.style={font=\sffamily\small},thin]
    (A) edge (F)
        edge (B)
    (B) edge (C)
        edge (D)
        edge (F)
    (C) edge (D)
    (D) edge (E)
    (E) edge (A)
    (F) edge (C)
        edge (D)
        edge (E);


\end{tikzpicture}

\column{.55\textwidth}
\begin{table}[h]
\begin{tabular}{|r|c|c|c|c|c|c|}
\hline
 & \multicolumn{6}{c|}{Distance $\delta (\_,J)$ to $J$} \\ 
\hline
$\delta (I,\_)$ & A &  B& C & D & E  & F \\
\hline
A & 0 & 1 & $\infty$ & $\infty$ & 1 & $\infty$ \\
\hline
B & $\infty$ & 0 & $\infty$ & $\infty$ & $\infty$ & $\infty$ \\
\hline
C & $\infty$ & $\infty$ & 0 & $\infty$ & $\infty$ & $\infty$ \\
\hline
D & $\infty$ & 1 & 1 & 0 & 1 & $\infty$ \\
\hline
E & $\infty$ & $\infty$ & $\infty$ & $\infty$ &0  & $\infty$ \\
\hline
F & $\infty$ & $\infty$ & $\infty$ & $\infty$ & $\infty$ & 0 \\
\hline
\end{tabular}
\end{table}
\end{columns}
That $F$ is off is reflected in the infinite distances.
\end{frame}


\begin{frame}{Illustration : B-F Convergence}
Next suppose process runs at $C$ and updates its own links and info from $D$. 
\begin{columns}
\column{.45\textwidth}
\begin{tikzpicture}[>=stealth',shorten >=2pt,auto,node distance=1.5cm,
  thin,main node/.style={circle,fill=blue!20,draw,font=\sffamily \bfseries}]


  \node[main node] (A) {A};
  \node[main node] (B) [above right of=A] {B};
  \node[main node] (F) [below right of=B] {F};
  \node[main node] (C) [above right of=F] {C};
  \node[main node] (D) [below right of=C] {D};
  \node[main node] (E) [below of=F] {E};

  \path[every node/.style={font=\sffamily\small},thin]
    (A) edge (F)
        edge (B)
    (B) edge (C)
        edge (D)
        edge (F)
    (C) edge (D)
    (D) edge (E)
    (E) edge (A)
    (F) edge (C)
        edge (D)
        edge (E);
\end{tikzpicture}

\column{.55\textwidth}
\begin{table}[h]
\begin{tabular}{|r|c|c|c|c|c|c|}
\hline
 & \multicolumn{6}{c|}{Distance $\delta (\_,J)$ to $J$} \\ 
\hline
$\delta (I,\_)$ & A &  B& C & D & E  & F \\
\hline
A & 0 & 1 & $\infty$ & $\infty$ & 1 & $\infty$ \\
\hline
B & $\infty$ & 0 & $\infty$ & $\infty$ & $\infty$ & $\infty$ \\
\hline
C & $\infty$ & 1 & 0 & 1 & 2 & $\infty$ \\
\hline
D & $\infty$ & 1 & 1 & 0 & 1 & $\infty$ \\
\hline
E & $\infty$ & $\infty$ & $\infty$ & $\infty$ &0  & $\infty$ \\
\hline
F & $\infty$ & $\infty$ & $\infty$ & $\infty$ & $\infty$ & 0 \\
\hline
\end{tabular}
\end{table}
\end{columns}
Distance from $C$ to $E$ is updated.
\end{frame}



\begin{frame}{Illustration : B-F Convergence}
Now suppose $B$ and $E$ update.
\begin{columns}
\column{.45\textwidth}
\begin{tikzpicture}[>=stealth',shorten >=2pt,auto,node distance=1.5cm,
  thin,main node/.style={circle,fill=blue!20,draw,font=\sffamily \bfseries}]


  \node[main node] (A) {A};
  \node[main node] (B) [above right of=A] {B};
  \node[main node] (F) [below right of=B] {F};
  \node[main node] (C) [above right of=F] {C};
  \node[main node] (D) [below right of=C] {D};
  \node[main node] (E) [below of=F] {E};

  \path[every node/.style={font=\sffamily\small},thin]
    (A) edge (F)
        edge (B)
    (B) edge (C)
        edge (D)
        edge (F)
    (C) edge (D)
    (D) edge (E)
    (E) edge (A)
    (F) edge (C)
        edge (D)
        edge (E);


\end{tikzpicture}

\column{.55\textwidth}
\begin{table}[h]
\begin{tabular}{|r|c|c|c|c|c|c|}
\hline
 & \multicolumn{6}{c|}{Distance $\delta (\_,J)$ to $J$} \\ 
\hline
$\delta (I,\_)$ & A &  B& C & D & E  & F \\
\hline
A & 0 & 1 & $\infty$ & $\infty$ & 1 & $\infty$ \\
\hline
B & 1 & 0 & 1 & 1 & 2 & $\infty$ \\
\hline
C & $\infty$ & 1 & 0 & 1 & 2 & $\infty$ \\
\hline
D & $\infty$ & 1 & 1 & 0 & 1 & $\infty$ \\
\hline
E & 1 & 2 & 2 & 1 &0  & $\infty$ \\
\hline
F & $\infty$ & $\infty$ & $\infty$ & $\infty$ & $\infty$ & 0 \\
\hline
\end{tabular}
\end{table}
\end{columns}
This may happen in different order due to asynchronity.
\end{frame}



\begin{frame}{Illustration : B-F Convergence}
Now let $A, C, D$ all update in some order.
\begin{columns}
\column{.45\textwidth}
\begin{tikzpicture}[>=stealth',shorten >=2pt,auto,node distance=1.5cm,
  thin,main node/.style={circle,fill=blue!20,draw,font=\sffamily \bfseries}]


  \node[main node] (A) {A};
  \node[main node] (B) [above right of=A] {B};
  \node[main node] (F) [below right of=B] {F};
  \node[main node] (C) [above right of=F] {C};
  \node[main node] (D) [below right of=C] {D};
  \node[main node] (E) [below of=F] {E};

  \path[every node/.style={font=\sffamily\small},thin]
    (A) edge (F)
        edge (B)
    (B) edge (C)
        edge (D)
        edge (F)
    (C) edge (D)
    (D) edge (E)
    (E) edge (A)
    (F) edge (C)
        edge (D)
        edge (E);


\end{tikzpicture}

\column{.55\textwidth}
\begin{table}[h]
\begin{tabular}{|r|c|c|c|c|c|c|}
\hline
 & \multicolumn{6}{c|}{Distance $\delta (\_,J)$ to $J$} \\ 
\hline
$\delta (I,\_)$ & A &  B& C & D & E  & F \\
\hline
A & 0 & 1 & 2 & 2 & 1 & $\infty$ \\
\hline
B & 1 & 0 & 1 & 1 & 2 & $\infty$ \\
\hline
C & 2 & 1 & 0 & 1 & 2 & $\infty$ \\
\hline
D & 2 & 1 & 1 & 0 & 1 & $\infty$ \\
\hline
E & 1 & 2 & 2 & 1 &0  & $\infty$ \\
\hline
F & $\infty$ & $\infty$ & $\infty$ & $\infty$ & $\infty$ & 0 \\
\hline
\end{tabular}
\end{table}
\end{columns}
Distances have converged to the best ones except those involving $F$.
\end{frame}



\begin{frame}{Illustration : B-F Convergence}
Now let $F$ come up.
\begin{columns}
\column{.45\textwidth}
\begin{tikzpicture}[>=stealth',shorten >=2pt,auto,node distance=1.5cm,
  thin,main node/.style={circle,fill=blue!20,draw,font=\sffamily \bfseries}]


  \node[main node] (A) {A};
  \node[main node] (B) [above right of=A] {B};
  \node[main node] (F) [below right of=B] {F};
  \node[main node] (C) [above right of=F] {C};
  \node[main node] (D) [below right of=C] {D};
  \node[main node] (E) [below of=F] {E};

  \path[every node/.style={font=\sffamily\small},thin]
    (A) edge (F)
        edge (B)
    (B) edge (C)
        edge (D)
        edge (F)
    (C) edge (D)
    (D) edge (E)
    (E) edge (A)
    (F) edge (C)
        edge (D)
        edge (E);


\end{tikzpicture}

\column{.55\textwidth}
\begin{table}[h]
\begin{tabular}{|r|c|c|c|c|c|c|}
\hline
 & \multicolumn{6}{c|}{Distance $\delta (\_,J)$ to $J$} \\ 
\hline
$\delta (I,\_)$ & A &  B& C & D & E  & F \\
\hline
A & 0 & 1 & 2 & 2 & 1 & $\infty$ \\
\hline
B & 1 & 0 & 1 & 1 & 2 & $\infty$ \\
\hline
C & 2 & 1 & 0 & 1 & 2 & $\infty$ \\
\hline
D & 2 & 1 & 1 & 0 & 1 & $\infty$ \\
\hline
E & 1 & 2 & 2 & 1 &0  & $\infty$ \\
\hline
F & 1 & 1 & 1 & 1 & 1 & 0 \\
\hline
\end{tabular}
\end{table}
\end{columns}
Now it will spread to all other nodes.
\end{frame}


\begin{frame}{Illustration : B-F Convergence}
Final convergence
\begin{columns}
\column{.45\textwidth}
\begin{tikzpicture}[>=stealth',shorten >=2pt,auto,node distance=1.5cm,
  thin,main node/.style={circle,fill=blue!20,draw,font=\sffamily \bfseries}]


  \node[main node] (A) {A};
  \node[main node] (B) [above right of=A] {B};
  \node[main node] (F) [below right of=B] {F};
  \node[main node] (C) [above right of=F] {C};
  \node[main node] (D) [below right of=C] {D};
  \node[main node] (E) [below of=F] {E};

  \path[every node/.style={font=\sffamily\small},thin]
    (A) edge (F)
        edge (B)
    (B) edge (C)
        edge (D)
        edge (F)
    (C) edge (D)
    (D) edge (E)
    (E) edge (A)
    (F) edge (C)
        edge (D)
        edge (E);


\end{tikzpicture}

\column{.55\textwidth}
\begin{table}[h]
\begin{tabular}{|r|c|c|c|c|c|c|}
\hline
 & \multicolumn{6}{c|}{Distance $\delta (\_,J)$ to $J$} \\ 
\hline
$\delta (I,\_)$ & A &  B& C & D & E  & F \\
\hline
A & 0 & 1 & 2 & 2 & 1 & 1 \\
\hline
B & 1 & 0 & 1 & 1 & 2 & 1 \\
\hline
C & 2 & 1 & 0 & 1 & 2 & 1 \\
\hline
D & 2 & 1 & 1 & 0 & 1 & 1 \\
\hline
E & 1 & 2 & 2 & 1 &0  & 1 \\
\hline
F & 1 & 1 & 1 & 1 & 1 & 0 \\
\hline
\end{tabular}
\end{table}
\end{columns}
\end{frame}



\begin{frame}{Illustration : Counting to Infinity}

Let us suppose there are three nodes $1, 2, 3$ and two nodes such that $\delta (1,2) = 1, \delta (2,3) = 1$. Asynchronous Bellman-Ford converges $\delta (1,3) = 1+1=2$.
Now suppose the link between nodes $2$ and $3$ goes off, or node $3$ goes off, and the following ensues at nodes $1$ and $2$:
\begin{itemize}
\item Distances are: $\delta (1,2) = 1$, $\delta (1,3)=2$ and $\delta (2,3)=\infty$.
\item Node $2$ thinks the least distance path to $3$ is through $1$ and resets $\delta (2,3)$ to $\delta (1,2) + \delta (1,3) = 3$.
\item Node $1$ thinks the least distance path to $3$ is through $2$ and resets $\delta (1,3)$ to $\delta (1,2) + \delta (2,3) = 4$.
\item Node $2$ thinks the least distance path to $3$ is through $1$ and resets $\delta (2,3)$ to $\delta (1,2) + \delta (1,3) = 5$.
\item Next, resets $\delta (1,3)$ to $6$, etc.
\end{itemize}

\end{frame}


